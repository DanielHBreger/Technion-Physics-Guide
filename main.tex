% !TEX program = xelatex

%%%%%%%%%%%%%%%%%%%%%%%%%%%%%%%%
%
% This is a template for a hebrew document in latex. 
% It is to be used to typeset lovely documents in the holey language
%
% version 1.0
%
% License: CC BY 4.0 
%
% Author: Elad Denenberg
%%%%%%%%%%%%%%%%%%%%%%%%%%%%%%%%%

\documentclass[a4paper,12pt]{article} 
%\usepackage[utf8]{inputenc}
\usepackage{parskip}
\usepackage[colorinlistoftodos]{todonotes}
\usepackage{hyperref}
\usepackage{polyglossia} %This is the package that gives access to fonts.
\setdefaultlanguage{hebrew}  %this makes the titles RTL, switch these if you write mostly in English
\setotherlanguage{english}  
\newfontfamily\hebrewfont[Script=Hebrew]{Arimo} %This font will work in Overleaf, Linux and Mac installs, if you're running windows you'd need to pick the right ones
\newfontfamily\hebrewfontsf[Script=Hebrew]{Miriam CLM}
\newfontfamily{\englishfont}{Noto Serif}

 %notice the \textenglish command here. This is because otherwise the RTL would screw up the parenthesis 
\title{כל מה שרציתם לדעת על ללמוד פיזיקה בטכניון ולא ידעתם מי לשאול\\
        \large גרסה 1.1}
\author{דניאל ברגר וסטודנטים לפיזיקה\footnote{תודה לאלון אפרת, אמיר פק, ברקוזילה, מר פרץ, סופי גליאטמן, עדן הרטמן, עידו סיטון, עידן בלייכר, שי אברהם, וסטודנטים שבחרו להשאר אנונימיים}}
\date{21.10.2025}

\begin{document}

\maketitle
\tableofcontents

\newpage

\section{מה זה המסמך הזה?}

ברוכים הבאים לפקולטה! מהחוויות של כמה מחזורים בשנים האחרונות הגענו למסקנה שצריך ליצור מסמך שמאגד את כל הידע שיש בפקולטה ומסתובב מפה לאוזן. יותר מדי פעמים אנשים לא ידעו כל מיני דברים חשובים וגילו רק בדיעבד מה לעשות. בתקווה המסמך הזה יכיל את כל מה שאתם צריכים לדעת, ואם לא, הוא יעזור לכם למצוא עם מי צריך לדבר בשביל להשיג את המידע.

המסמך נכתב בלעטך והוא וקוד המקור שלו זמינים ב 

\textenglish{\url{https://github.com/DanielHBreger/Technion-Physics-Guide}}.

במידה ואתם רוצים להוסיף מידע למסמך לגרסאות חדשות אתם מוזמנים לשלוח אותו אליי, או אם אתם יודעים להשתמש בגיטהאב אתם מוזמנים לפתוח \textenglish{issue} או \textenglish{pull request}. ככל הנראה יצאו גרסאות מעודכנות עם הזמן לכן מומלץ להיכנס לקישור לבדוק פעם בכמה זמן.

\section{יש לי בעיה - מי יכול לעזור?}

בפקולטה יש הרבה גורמים שיודעים לעזור בכל מיני סוגים של בעיות, ומאוד כדאי להכיר אותם:

\begin{enumerate}
    \item אם יש בעיה אקדמית בקורס שאתם לוקחים, הכתובת הראשונה היא נציג הקורס או נציג הסמסטר. הנציגים יודעים למי לפנות ואיך לטפל בבעיות השונות, ואם הם לא מצליחים, הם יודעים למי להעביר הלאה.
    \item הרכז האקדמי של הפקולטה מטעם אס"ט הוא הדרג מעל הנציגים, אם יש לכם בעיה שלא לנציג אבל אתם רוצים להתייעץ עם מישהו שלא איש סגל, הוא הכתובת.
    \item באופן כללי אם יש לכם בעיה שלא לשני הגורמים לעיל הכתובת הראשונה בפקולטה היא אביטל רכזת ההסמכה
    \item אם יש לכם בעיה כלשהי בקשר למילואים - דברו עם חנה נצר בלימודי הסמכה (תודה לאנונימי)
    \item אם יש לכם בעיה אחרת, קודם כל אפשר לפנות גם לפקולטה דרך אביטל או אחד מסגני הדיקן, אבל אפשר גם לפנות אל לשכת דיקן הסטודנטים.
    \item \href{https://dean.technion.ac.il/}{לשכת דיקן הסטודנטים} מציעה כל מיני סוגים של עזרה, מכלכלית עם מלגות עד עזרה בראיונות עבודה והפניה לשירותים פסיכולוגיים
\end{enumerate}

.אפשר למצוא את כל הדרכים ליצור קשר עם כל הסגל באתר הפקולטה \href{https://phys.technion.ac.il/he/undergraduatepage/underg-contact}{(קישור)}

\subsection{מה אם נכשלתי בקדם?}

מבחינת הפקולטה יש מדיניות שאם קיבלת בקורס של פיזיקה 45 אתה יכול להמשיך לקורסי המשך שלו (עדיין תצטרך לחזור על הקורס אם קיבלתם פחות מ55) (תודה לאנונימי)

באופן כללי בטכניון אם עברתם בציון הסופי מועד אחד לפחות (נניח עברתם במועד א' ונכשלתם במועד ב') ניתן לקחת את הקורסי המשך, אם נכשלתם במועד האחרון תצטרכו לחזור על הקורס. (תדה לאנונימי)

\subsection{קיבלתי אזהרה שאני לא תקין אקדמית!}

קודם כל לנשום - הכל בסדר. אי תקינות אקדמית זה נטו דרך של הטכניון להוציא כרטיס צהוב למי שמתקשה עם הלימודים. בתכלס מה שזה אומר זה שתהיה לכם שיחה עם סגן הדיקן שבה ינסו להבין למה היה לכם קשה עם הלימודים ואם יש להם דרך לעזור לכם. תקבלו כנראה תכנית מחייבת, שזה פשוט סט של קורסים שאומרים לכם לקחת ולעבור בציון מינימלי כלשהו. רוב הסיכויים גם תוזמנו לפגישה עם היועצים של המרכז לקידום סטודנטים, תנצלו את זה, הם באמת רוצים לעזור ויש להם כלים טובים. יש לכם גם את המלווה באי תקינות של אס"ט לדבר איתו.

\section{דברים כלליים שכדאי להכיר}

\emph{למערכת המומלצת יש שם מתעתע. אל תרגישו מחוייבים לעקוב אחרי המומלצת, לפעמים לסטות ממנה זה הדבר הנכון לעשות. הדבר היחיד שצריך לדעת זה שאם חורגים ממנה ביותר משנתיים נהיים לא תקינים אקדמית, אבל זה לא סוף העולם. שימו לב רק שאתם לא מזיזים קורסים שהם קדמים אחד של השני ושבכללי נשמר ההיגיון בסדר של הקורסים.}

אל תבהלו משיעורי בית בלתי נגמרים, ותדעו מתי כדאי לוותר על חלק מהם או לא לרפרנס בשביל להשקיע את הזמן בלהבין את החומר יותר טוב במקום. לפעמים להצליח פחות בגיליונות ולקבל בחלק מהם 70 יעזור לכם לדעת את החומר טוב יותר למבחן ואז זה לא משנה בכל מקרה.

אל תפחדו לנסות מרצים ומתרגלים אחרים ממי שהתחלתם את הקורס איתו. לא תמיד מתחברים לכולם וזה בסדר.

\textbf{תנצלו את שעות הקבלה זה הכלי הכי טוב שיש לכם בטכניון.}

תביאו בגד חם ומחשבון לכל המבחנים (אלא אם אסור). עדיף שיהיה ליתר בטחון. (\emph{תודה לברקוזילה})

ברוב הדברים שאתם משתמשים בהם יש הטבות לסטודנטים. בין אם זה הנחה או אפילו היכולת להשתמש חינם לתקופה. תחקרו קצת על זה, זה שווה.

\subsection{פייתון לפיזיקאים}\label{pythonlearn}

\textbf{במהלך הקורס פייתון תעשו מה שמבקשים ממכם בקורס, תחזרו לפה אחרי זה.} הקורס מבוא למדמח בפייתון מלמד בעיקר מבוא למדעי המחשב ובגישה ספציפית ולדעתי יש גישה יותר נוחה. המטרה של החלק הזה היא לעזור לכם להגדיר פייתון בצורה שתהיה לכם נוח לעבוד איתה ולעזור לכם ללמוד פייתון. 

\subsubsection{התקנות ראשוניות - למי שלא התחבר לג'ופיטר}

במחשב שלכם אתם לא חייבים להשתמש בכל המערכת שהטכניון מלמד להשתמש בה עם אנקונדה וג'ופיטר, אני ממליץ להשתמש ב\textenglish{visual studio code}.

\begin{enumerate}
    \item תוודאו שמותקן אצלכם פייתון הכי עדכני
    \item תתקינו את \textenglish{Visual Studio Code} מפה: https://code.visualstudio.com/
    \item בתוך \textenglish{VSC} תלכו ל\textenglish{Extensions} (האייקן בצורת ארבעה ריבועים משמאל של המסך)
    \item תחפשו ותתקינו את: \begin{enumerate}
        \item \textenglish{Jupyter}
        \item \textenglish{Python}
        \item \textenglish{Pylance}
    \end{enumerate}
\end{enumerate}

ועכשיו כל פעם שאתם עובדים עם פייתון תשתמשו ב \textenglish{vscode}. הוא עובד גם עם קבצי פייתון רגילים וגם עם קבצי \textenglish{ipynb}.

\subsubsection{ללמוד אשכרה פייתון}

רשימת מקורות לפי רמת הידע הקיים שלכם:

\begin{enumerate}
    \item אני מעולם לא תכנתתי - קורס המבוא לתכנות של אוניברסיטת הלסינקי: 
    
    \textenglish{https://programming-25.mooc.fi/part-1/1-getting-started}.  
    \item אני לא מכיר את פייתון אבל תכנתתי בשפה אחרת, ואני רוצה הסבר מודרך - קורס הפייתון של \textenglish{W3schools}: 
    
    \textenglish{https://www.w3schools.com/python/python\_intro.asp}
    \item אני רק צריך שתראה לי את ה\textenglish{syntax} ודוגמאות, כבר אבין לבד - 
    
    \textenglish{https://learnxinyminutes.com/python/}
\end{enumerate}

\subsubsection{ספריות פייתון שימושיות שכדאי להכיר}

במעבדות ובכמה קורסי חובה תשתמשו בקוד פייתון שנסמך על כמה ספריות פייתון פופלריות: \textenglish{Scipy, Numpy, Matplotlib}. ייתנו לכם במעבדה 2 למידה בסיסית ובמעבדה 3 סדנת פייתון לעשות, אבל לבוא מוכנים מראש זה רק טוב.

\textenglish{Numpy: https://www.w3schools.com/python/numpy/default.asp}

\textenglish{Scipy: https://www.w3schools.com/python/scipy/index.php}

\textenglish{matplotlib: https://www.w3schools.com/python/matplotlib\_intro.asp}

\subsection{המעבדות}

תעברו על תדריך הניסוי לפני שאתם מגיעים למעבדה. במעבדה 1 זה עוד לא קריטי, אבל הניסויים מסתבכים מהר במעבדות הבאות ולהגיע לא מוכן לא נעים (\emph{תודה למר פרץ})

כדאי למצוא שותף שאתם מכירים שאתם יכולים לעבוד איתו מראש. שותף מוצלח במעבדה יעזור לכם במהלך כל הסמסטר, והמעבדות שותות לכם את הזמן, אז זה חשוב. (\emph{תודה למר פרץ})

במעבדה 3 כבר יתחילו לצפות שתכתבו דוחות מעבדה כמו שצריך, ואתם לא רוצים לכתוב דוחות מעבדה בוורד. תלמדו לכתוב בלעטך.

הטכניון מספק לכולם חשבון ל\textenglish{https://www.overleaf.com/} ויש להם גם מדריך טוב ללעטך:
\textenglish{https://www.overleaf.com/learn}

חלק מהמסלולים, בעיקר אלה שאין להם מעבדות אחרות בנוסף, צריכים לעשות פרויקט במעבדה 2 או 3 (השם של הקורסים הוא 2מפ או 3מפ בהתאמה). כמעט כל מי שעושה את הפרויקטים האלה אומר שזה הדבר הכי כיף שעשו בתואר עד אז, אז אין מה לפחד.

\subsection{הקלות והתאמות}

הטכניון בד"כ מקבל רק הקלות של מאל"ו או של רמב"ם, או הקלות רפואיות, ז"א מרופא או פסיכיאטר. הקלות שמקבלים ממאל"ו זה מול רכזי הקלות לימודים. הקלות רפואיות זה מול רכזת התאמות רפואיות וזה מותנה בועדה רפואית. בתכלס, ברגע שיש הקלה צריך לשלוח אותה במייל ואו מקבלים אישור או שקובעים ועדה, כל סמסטר יש תאריך אחרון לשלוח בשביל שזה ייחשב גם לסמסטר הנוכחי.

\subsection{מלגות}

יש מלגות של דיקן הסטודנטים, אבל יש גם כל מיני מלגות שלא קשורות אליו אבל די קבועות בטכניון כמו אנייר.

\subsection{קישורים שימושיים}

\begin{enumerate}
    \item \href{https://cheesefork.cf/}{צ'יזפורק}
    \item \href{https://chromewebstore.google.com/detail/pfhjnidbfndnjhpcpfecngcigdjebemk}{טכניון++}
    \item \href{https://tscans.cf/}{מאגר הסריקות}
    \item \href{https://drive.google.com/drive/folders/1g5687tI9s-fwjgbp4C28qpOQerYVEQVp}{הדרייב הפקולטי}
    \item \href{https://www.admin.technion.ac.il/dpcalendar/}{לוח השנה הטכניוני הכי מפורט}
    \item \href{https://portalex.technion.ac.il/irj/portal/external}{הסאפ}
    \item \href{https://grades.technion.ac.il/index.aspx}{מערכת הציונים}
    \item \href{https://www.facebook.com/groups/118649778226975}{קבוצת הפייסבוק סטודנטים בטכניון}
    \item \href{https://www.desmos.com/}{דסמוס - מחשבון גרפי}
    \item \href{https://chromewebstore.google.com/detail/eclmfdfimjhkmjglgdldedokjaemjfjp}{תוספת לדסמוס שמשפרת אותו עם מלא פיצ'רים}
\end{enumerate}

\section{המסלולים}

המסלולים השונים פותחים לכם אפשרויות שונות, אבל כל מסלול גם כולל פשרות משלו ודברים אחרים שכדאי לדעת לגביהם.

\subsection{פיזיקה טהור}

תלוי באיזה קטלוג אתם מסתכלים עלולה להיות טעות בסכימת הנקודות שצריך לתואר. המספר הנכון הוא 119.5 נק"ז סך הכל, מתוכם 92.5 חובה, 17 בחירה מפיזיקה, 2 בחירה חופשית, 2 ספורט, ו6 מל"ג.

הסמסטר הכי קשה בתואר הוא סמסטר 3, לא כי הקורסים בהכרח קשים אפילו, פשוט יש 7 קורסים וקשה לעקוב אחרי מה שהולך ותקופת המבחנים מאוד עמוסה.

אל תילחצו על הציונים של הקורסים 1פ ו2פ. ברגע שיש לכם ציון באנליטית ובאלקטרו הציון בקורסים האלה נהיה פחות חשוב.

ממפי"ס לא קריטי אבל מאוד נחמד לפיזיקה 2פ. זה בגדול קורס סופר-מזורז בשיטות בחדוא 2ת + מדר וזה עושה את החצי הראשון של סמסטר 2 הרבה פחות מבלבל. מקבלים בסוף עובר בינארי וצריך איזה 40 בשביל לעבור אז גם לא צריך להשקיע, רק את המינימום, ולידיה באמת ממש חמודה בקורס וממש משקיעה.

מאוד מומלץ לעשות פרויקט במקום מעבדה 6, גם כי פשוט יימאס לכם ממעבדות עד אז, וגם כי זה נותן לכם ניסיון בלעשות משהו שיותר קרוב לאיך שעושים פיזיקה בפקולטה באמת. בשביל לעשות פרויקט פשוט הולכים ומדברים עם פרופסורים בפקולטה לגבי המחקר שלהם ושואלים אם יש להם פרויקטים לתת. יש שני פרויקטים, של 3 נק"ז ושל 4.5. פרויקטים של 4.5 נק"ז הם לרוב פרויקטים באורך של שנה שלמה, אז אם אתם מתכננים לעשות פרויקט כזה, תתחילו להתייעץ עם אנשי סגל כבר בסמסטר 4. יש פרופסורים שנותנים רק פרויקטים של 4.5 נק"ז, ויש גם פרופסורים שמציבים דרישות של ממוצע מינימלי או התחייבות ללמוד קורסים שונים בתור תנאי לעשות אצלם פרויקט.

\subsection{פיזיקה-מתמטיקה}

(תודה לשי אברהם על הסעיף)

הייתי ממליץ לעשות מד"ח לפני או במקביל לאלקטרו כי הוא ממש מסדר את החלק השני של אלקטרו (שזה פשוט כל מיני שיטות שאת רובן רואים במד"ח לאיך לפתור את משוואות מקסוול). כנ"ל שיטות אנליטיות של לידיה. הקורס הזה עשה מלא סדר באלקטרו ואם אפשר לקחת אותו לפני או במקביל לאלקטרו זה יהיה מדהים.

בנוסף, אני מאוד ממליץ לקחת מבוא לשימושית. מעבר לזה שזה אחלה בחירה, אני חושב שזה הקורס שהכי משקף איך מתמטיקאים מתמודדים עם מודלים פיזיקליים והוא יכול לתת תובנות שיעזרו לתואר בפיזיקה.

לאינפי 1/2 חייבים לקרוא את הרשימות של רון רוזנטל. הן מאוד מסודרות וריגורוזיות, ואלה שני הקורסים שהכי חשוב לשים לב לריגורוזיות כי אלה הקורסים שבודקים את זה. בקורסים מתקדמים יותר כבר מניחים שיש לך בגרות מתמטית אז לא יקטלו אותך על כל דבר קטן. אני עדיין זוכר שבמבחן באינפי 1 הורידו לי נקודה כי לא נימקתי למה מותר לי לקחת שורש של z למרות שהיה נתון ש-z חיובי. באינפי 1 חייבים לנמק להם הכל.

\subsection{פיזיקה-הנדסת חשמל}

(תודה לסטודנט אנונימי על הסעיף)

בסמסטר א יש קורסי בחירה שאפשר לקחת רק בשנה הקרשונה כמו מבט על הנדסת חשמל, פרוייקט מבוא, וממפיס. אני ממליצה מאוד על הפרויקט כי החלק הראשון של הסמסטר זה פשוט הרצאות של הסגל של הפקולטה לחשמל, שנותנים טיפים, עושים סיורים במעבדות, והחלק של הפרויקט הוא רק מצגות, אז זה לא כזה קשה בסוף מציגים גם לכמה אנשי סגל, אז זה מגניב. אבל מה שהכי חשוב לזכור בקורסים האלה זה שהם נוכחות חובה (חוץ ממפיס) אז צריך לבדוק שזה לא נופל על הרצאה או תרגול חשוב.

לגבי מעבדות, במסלול הרגיל (חשמל פיזיקה משולב) יש 5 מעבדות של פיזיקה ועוד 2 של חשמל. עדיף לעשות את המעבדות 1 ו2 של פיזיקה בשנה א', כי שנה ב מאוד עמוסה (סמסטר 3 יש 7 קורסים, ובסמסטר 4 יש 3 או 4 קורסים של 5 נקודות). המעבדה של חשמל 1א ממש כבדה וצריך להשקיע הרבה זמן אז קשה להכניס אותה בשנה ב וגם בדרך כלל קשה לתפוס מקום במעבדה (חוץ מעתודאים כי יש להם עדיפות).

לגבי סמסטר 4, יש בחירה בין אלקטרו לשדות. אם תרצו לעשות תואר שני בפיזיקה אז חייב את אלקטרו. זה כן יותר עבודה אבל ב2 הקורסים הציונים לא משהו. אם אתם בטוחים שאתם לא רוצים תואר שני בפיזיקה אז תקחו שדות.

לגבי המומלצת, יש יותר מ20 נקז של קורסי חובה בכל סמסטר עד סמסטר 5 שזה מאוד עמוס. אם תרצו לדחות קורס תזכרו שכל קורסי החובה של פיזיקה הם חד שנתיים. כלומר אם אתם רוצים לדחות קורס של פיזיקה רק תוכלו לקחת אותו בעוד 2 סמסטרים ולא בסמסטר הבא. זה גם אומר שכל הקורסים שצריכים אותו כקדם ידחו בשנה. לעומת זאת, הקורסים חובה של חשמל ניתנים כל סמסטר אז אם רוצים לדחות או לחזור על קורס אפשר לקחת אותו בסמסטר שאחרי. אם תרצו לדחות קורס מומלץ לבדוק בדיוק למה הוא קדם ולמה הקורסים הבאים הם קדם. לחזור על קורס קדם זה קצת שונה כי לפעמים אפשר לקחת את הקורס הבא במקביל. (לדוגמה: אם נכשלים במבוא למדמח, אפשר לבקש לעשות בסמסטר 2 את ספרתיות אבל אם הורדתם מדמח בסמסטר א אז לא תוכלו לעשות אותם במקביל)

\subsection{פיזיקה-מדעי המחשב}

(תודה לאנונימי על הסעיף)

מומלץ לקחת את מד"ר ת במקום מד"ר א למי שלא רוצה להתעמק במד"ר כמו מתמטיקאי, הממוצע גבוה יותר ולומדים שם את הפרקטיקה של איך לפתור מד"ר ולא את המסביב.

ספציפית בפיזיקה-מדמ"ח מומלץ כן לעקוב אחרי המומלצת ולא לגלוש ממנה. אם אתם יכולים להקדים קורס אחד מסמסטר 3 לסמסטר 2 זה מאוד יעזור לכם כי סמסטר 3 מאוד עמוס. הכי מומלץ להקדים את אלגברה מודרנית או אנליזה וקטורית.

אל תדאגו מהציון של מתמ גם אם אומרים לכם שהוא חשוב. מבני ואלגוריתמים הם חשובים יותר אבל אל תקרעו את עצמכם.

\subsection{פיזיקה-הנדסה ביורפואית}

(תודה לאנונימי על הסעיף)

נכון לשנת 2025, יש (בכל השנים) כ15 סטודנטים סך הכל שעושים את התואר המשולב בהנדסה ביו-רפואית ופיזיקה. למה? כי הרבה אנשים מוותרים על אחד מהתארים בשנים הראשונות. אם אתם לא מעוניינים להיות בין האנשים האלה הנה כמה עצות:

\begin{enumerate}
    \item יכול להיות שהתואר המשולב לא בשבילכם - 
התואר בביו-רפואה הוא פחות מתמטי ועוסק הרבה בשינון ולמידת עקרונות מדעיים אבסטרקטיים. לעומת חשמל/אוירו/מדמח, אשר גם בלי פיזיקה מתבססים הרבה על מתמטיקה, (ולכן השילוב עם פיזיקה נותן ערך נוסף כלשהו גם לתואר ההנדסי), בביו-רפואה זה פחות המצב. אל תצפו ש"לשלב פיזיקה יהפוך אתכם למהנדסי ביו-רפואה יותר טובים", כי זה לא. אם החלק המתמטי הוא החלק שאתם פחות נמשכים אליו בתואר, כנראה שהשילוב לא עבורכם.
    \item ה"מומלצת" -
    ה"מומלצת" של התואר המשולב היא בעייתית בלשון המעטה. סמסטר ג' (אשר ידוע לשמצה בכל התארים) הוא מאתגר מאוד, ויש קורסים כבדים מאוד שנדחים לסוף התואר (מתי שאתם לא רוצים קורסים כבדים שאם תיכשלו בהם יעכבו לכם את סגירת התואר). בגלל זה, נפוץ לראות אנשים שמקדימים/מאחרים קורסים, ולא צריך להילחץ אם רואים מישהו שנה מעליכם/מתחתיכם באותו הקורס (מעטים האנשים שדיברתי איתם שלא האריכו את התואר בסמסטר או שניים). המלצות שלי לשינויים במומלצת:
    \begin{itemize}
        \item להקדים את מעבדה 2 מסמסטר ג' לב'
        \item את שאר המעבדות (3 והלאה) לא לפחד לדחות אפילו בשנה
        \item להחליף את גלים באנליטית (סמסטר ג')
        \item להבין כמה שיותר מהר איזה מגמות רוצים- קורסים שהם לא קדם למגמות האלה יכולים סתם להוסיף ללחץ בסמסטרים הראשונים ואפשר לדחות אותם.
    \end{itemize}
    \item אם תישארו במומלצת יהיו לכם סמסטרים עמוסים ותקופות מבחנים לחוצות. אם תצאו מהמומלצת אולי אפילו יהיו לכם התנגשויות של מבחנים והרצאות. השורה התחתונה היא \textbf{אין מה לעשות}. כשיש מצבים של לחץ שווה לדעת לשחרר. לחוצים מהגשות של שיעורי בית - תבקשו רפרנס מחבר/מדרייב. יש תקופת מבחנים ארוכה - תגשו לחלק מהם רק במועד ב. אל תפחדו להיות לא "על החומר" כל הזמן (בעיקר בקורסים של ביו-רפואה!), ותנהלו את הסמסטר בצורה אסטרטגית - אי אפשר להיות הכי טובים בהכל כל הזמן, אבל אפשר לתכנן במה הכי חשוב להיות טובים עכשיו.
\end{enumerate}

בגדול זה שילוב קשה, והוא משלב המון תחומים שונים. לדעתי זה מדהים ללמוד פיזיקה וביו-רפואה ביחד וכל אחד מהמקצועות האלה מרתק ומביא איתו עולם שונה לחלוטין של חשיבה. תעזרו בחברים שלכם משילובים אחרים של הנדסה ופיזיקה, ויותר מהכל אל תתביישו להיעזר בסטודנטים שלומדים ביו-רפואה פיזיקה שנים מעליכם.

\subsection{פיזיקה-הנדסת אווירונאוטיקה וחלל}

(תודה לעידן בלייכר על הסעיף)

בזמן כתיבת המסמך ב2025, זה יהיה הסמסטר הראשון בו מחזור של סטודנטים יהפכו לבוגרי התואר המשולב, ככה שהתכנית עדיין חדשה. גם בגלל זה וגם בגלל שמדובר על שילוב עם תואר הנדסה, התכנית תהיה לפחות 4 שנים, וכנראה שלקיחת מקצועות בחירה יהיה טריקי עד שתתחילו את השנה הרביעית. זה עדיין אפשרי בשנה השלישית, וטכנית בשנה השנייה יש קורסי בחירה של פיסיקה שאפשר ללמוד., אבל המערכת לא תמיד תהיה לטובתכם.  

יש תמיכה ועזרה בשילוב עבודה מהפקולטה לאווירונאוטיקה, ויש חברות שישימו לב אליכם יותר אם אתם בתואר הדו-חוגי ועם ממוצע טוב (רפאל למשל). אפשר לשלב עבודה עם הלימודים, אבל ייתכו ותצטרכו לפרוש את התואר כדי לעמוד בעומס. 

\subsection{פיזיקה-הנדסת חומרים}

(תודה לאלון אפרת ועדן הרטמן על הסעיף)

המסלול משלב בין שתיים מהפקולטות המשפחתיות והתומכות ביותר בטכניון, ממליץ מאוד לנצל את זה ע"ע הצגת נוכחות במרחבי הלמידה.

בסמסטרים הראשונים רוב הקורסים שתיקחו יהיו קורסים של פיזיקה (שהם בד"כ יותר מאתגרים). קחו את זה לתשומת לבכם ותזכרו שזה משתפר ושאם קשה לכם זה בסדר.

סמסטר 3 קשה בעיקר כי הוא משלב המון קורסי מתמטיקה עם 2 מעבדות ולכן ממליץ לכם לקחת את המעבדה בחומרים ח' בקיץ. עם זאת, ממליץ להמנע מהוספת קורסים בשנה א' (להוריד זה בסדר). בנוסף בסמסטר א' מציעים שני קורסים של "נקז חינם", ממפיס ועקרונות. ממפיס עוזר להבין עקרונות מתמטיים שנחוצים להבנת הפיזיקה ועקרונות בעיקר נחמד כי הוא עוזר להכיר את התחומים שהפקולטה מתעסקת בהם ואת שאר המחזור.

\subsection{מסלולים מומצאים ותואר כפול}

(תודה לעידו סיטון על הסעיף)

\subsubsection{פיזיקה-הנדסת מכונות \textenglish{(Join at your own risk)}}

המסלול נכון לשנת 2026 אינו קיים אך עובדים על אישורו. נשארו "רק" ישיבת ועדות וסבב אישורים בירוקרטי - מעבר לכך כל התשתית הטכנית, כמו מומלצת לאישור 2 הפקולטות, קיימת. ולכן, במידה ואתם מעוניינים במסלול זה אני מקווה (ועושה את כל המאמצים) שהוא יפתח בזמנכם.

עצות: תנסו לדחוף מעבדות כמה שיותר מוקדם, תסיימו עם זה. מעבדה 1 של פיזיקה ממש ממש קלה ולא דורשת הרבה זמן מעבר ל3 שעות של המעבדה עצמה אז שהיא לא תהיה באמת שיקול מבחינתכם.

\subsubsection{תואר כפול}

במידה ואתם מעוניינים בתואר משולב שלא קיים ישנה חלופה: על פי תקנה 3.2.2 של הטכניון לאחר סיום 72 נק"ז מכלל התואר אליו אתם רשומים אתם רשאים לבקש בקשה לבצע תואר נוסף. הבקשה עוברת לאישור של הפקולטה אליה תרצו להצטרף ולאישור מלימודי הסמכה. לאחר קבלת האישורים הרלוונטיים תקבלו "תוכנית" (רשימה של קורסים שעליכם להשלים בשביל התואר הנוסף).
אפרט על חישוב הנק"ז של כלל התואר בנפרד.

\begin{enumerate}
    \item עצה שלי, אם אתם חושבים בשלב יחסית מוקדם ללכת לכיוון הזה תתחילו להסתכל על המומלצות של 2 התארים ולהוסיף לעצמכם קורסי חובה גם של התואר אליו תרצו להשתייך בהמשך. זה יקל עליכם על השלמות קורסים והארכת התואר
    \item טיפ חשוב: בבניית הלוז לסמסטר – חפיפה בין קורסים של התארים השונים מבחינת שעות שבועיות היא יותר מסבירה וזה בסדר, לא להיבהל, מאוד קל להשלים הרצאות או תרגולים פה ושם. הדגש צריך להיות על המבחנים. שימו לב שלוז המבחנים שלכם ריאלי ואפשרי (שאין יותר מידי מבחנים ביום אחד ושיש מרווחים טובים בין מבחנים של קורסים גדולים וקשים). במידה ולא, וקיימת האפשרות כמובן, חשבו על לדחות קורס בתקווה שבסמסטר אחר המבחנים יסתנכרנו בצורה טובה יותר. אם אתם רוצים לדעת תאריך של בחינה טרם פורסמה, אפשרי ללכת למזכירות של הפקולטות, לשאול ולהתייעץ.
    \item המזכירה היא החברה הכי טובה שלכם. היא תעזור לכם בכל מה שתצטרכו. אל תהססו לדבר איתה!
    \item דבר נוסף, ישנם קורסים שמבחינת סילבוס הם זהים. אל תצאו מנקודת הנחה שתקבלו עליהם פטור מיידי. קבעו שעת קבלה עם סגן הדיקן ובקשו ממנו פטור מהקורס תוך הצגת החפיפה בסילבוסים בין הקורס שעשיתם לקורס הנדרש. סיכוי גבוה שהוא יאשר אבל אל תיקחו זאת כמובן מאליו.
    \item בתחילת התואר ישנם קורסי מתמטיקה. בכל תואר רמה שונה של העמקה בכל תחום. חשוב שתיקחו את הקורס המעמיק יותר מבין המומלצות של התארים. לדוגמה, חדו"א 1מ כחובה למכונות וחדו"א 1ת כחובה לפיזיקה על פי המומלצת. כיוון שחדו"א 1ת מעמיק יותר ומכיל בתוכו את חדו"א 1מ, קחו מלכתחילה את חדו"א 1ת ותמנעו מבעיות מול סבב האישורים בבוא העת.
\end{enumerate}

חישוב נקז לתואר כפול: סכום הנקז של שני התארים ואז 70 אחוז מזה זה כמות הנקז הכוללת שאתם צריכים לצבור כדי לסיים את התואר הכפול. קודם אתם צריכים לסיים את הדרישות לפי הקטלוג של התואר אליו נרשמתם, כלומר כל קורס חופף יכנס לתואר אליו אתם רשומים ואם הוא אקסטרה נק"ז האקסטרה הולך לבחירות לפי מה שמוגדר לכם ברשימות בחירה. כל הנקז הנותר שזה ה70 אחוז מהסכום פחות הנק"ז הכולל של התואר אליו נרשמתם זה נק"ז שצריך להשלים רק מהתואר שנוסף. לשם נכנסים כל הקורסים שהם לא בנק"ז של התואר אליו אתם רשומים בכלל בין אם זה חובה ובין אם זה בחירה בתואר הנוסף.

\section{אחר}

יש אפשרות לשחק כדורגל וכדורסל במגרשים של הטכניון יש מערכת כזאת שפשוט צריך לתפוס בה מקום באתר של אס"א, בעמוד "פעילות חופשית" שמופיע תחת הקטגוריה "אירועים בקמפוס".

\end{document}

